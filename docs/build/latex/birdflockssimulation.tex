%% Generated by Sphinx.
\def\sphinxdocclass{report}
\documentclass[a4paper,11pt,oneside,english]{sphinxmanual}
\ifdefined\pdfpxdimen
   \let\sphinxpxdimen\pdfpxdimen\else\newdimen\sphinxpxdimen
\fi \sphinxpxdimen=.75bp\relax

\PassOptionsToPackage{warn}{textcomp}
\usepackage[utf8]{inputenc}
\ifdefined\DeclareUnicodeCharacter
% support both utf8 and utf8x syntaxes
  \ifdefined\DeclareUnicodeCharacterAsOptional
    \def\sphinxDUC#1{\DeclareUnicodeCharacter{"#1}}
  \else
    \let\sphinxDUC\DeclareUnicodeCharacter
  \fi
  \sphinxDUC{00A0}{\nobreakspace}
  \sphinxDUC{2500}{\sphinxunichar{2500}}
  \sphinxDUC{2502}{\sphinxunichar{2502}}
  \sphinxDUC{2514}{\sphinxunichar{2514}}
  \sphinxDUC{251C}{\sphinxunichar{251C}}
  \sphinxDUC{2572}{\textbackslash}
\fi
\usepackage{cmap}
\usepackage[T1]{fontenc}
\usepackage{amsmath,amssymb,amstext}
\usepackage{babel}



\usepackage{times}
\expandafter\ifx\csname T@LGR\endcsname\relax
\else
% LGR was declared as font encoding
  \substitutefont{LGR}{\rmdefault}{cmr}
  \substitutefont{LGR}{\sfdefault}{cmss}
  \substitutefont{LGR}{\ttdefault}{cmtt}
\fi
\expandafter\ifx\csname T@X2\endcsname\relax
  \expandafter\ifx\csname T@T2A\endcsname\relax
  \else
  % T2A was declared as font encoding
    \substitutefont{T2A}{\rmdefault}{cmr}
    \substitutefont{T2A}{\sfdefault}{cmss}
    \substitutefont{T2A}{\ttdefault}{cmtt}
  \fi
\else
% X2 was declared as font encoding
  \substitutefont{X2}{\rmdefault}{cmr}
  \substitutefont{X2}{\sfdefault}{cmss}
  \substitutefont{X2}{\ttdefault}{cmtt}
\fi


\usepackage[Bjarne]{fncychap}
\usepackage{sphinx}

\fvset{fontsize=\small}
\usepackage{geometry}


% Include hyperref last.
\usepackage{hyperref}
% Fix anchor placement for figures with captions.
\usepackage{hypcap}% it must be loaded after hyperref.
% Set up styles of URL: it should be placed after hyperref.
\urlstyle{same}


\usepackage{sphinxmessages}




\title{Bird Flocks Simulation\\
  \LARGE Documentation}
\date{Jun 07, 2021}
\release{}
\author{Anna Danot, Núria Fernández,\\ Jan Mousavi, Loredana Sandu}
\newcommand{\sphinxlogo}{\vbox{}}
\renewcommand{\releasename}{}
\makeindex


\begin{document}

\pagestyle{empty}
\sphinxmaketitle
\pagestyle{plain}
\sphinxtableofcontents
\pagestyle{normal}
\phantomsection\label{\detokenize{index::doc}}

\chapter{parameters module}
\label{\detokenize{parameters:module-parameters}}\label{\detokenize{parameters:parameters-module}}\label{\detokenize{parameters::doc}}\index{module@\spxentry{module}!parameters@\spxentry{parameters}}\index{parameters@\spxentry{parameters}!module@\spxentry{module}}\phantomsection\label{\detokenize{parameters:module-0}}\index{module@\spxentry{module}!parameters@\spxentry{parameters}}\index{parameters@\spxentry{parameters}!module@\spxentry{module}}
Parameters used while running the simulation.

\bigskip 
\bigskip 

\begin{fulllineitems}
\pysigline{\sphinxbfcode{\sphinxupquote{DIM:}}}
(\sphinxtitleref{int}) dimension of the container and simulation (2 for plane, 3 for cube).

\end{fulllineitems}



\begin{fulllineitems}
\pysigline{\sphinxbfcode{\sphinxupquote{NUM\_BIRDS:}}}
(\sphinxtitleref{int}) number of birds in simulation.

\end{fulllineitems}

\bigskip
\bigskip 

\begin{fulllineitems}
\pysigline{\sphinxbfcode{\sphinxupquote{ATTRACTION\_POINTS:}}}
(\sphinxtitleref{list}) coordinates where attraction points will be initially located.

\end{fulllineitems}



\begin{fulllineitems}
\pysigline{\sphinxbfcode{\sphinxupquote{REPULSION\_POINTS:}}}
(\sphinxtitleref{list}) of coordinates where repulsion points will be initially located.

\end{fulllineitems}


\bigskip
\bigskip 


\begin{fulllineitems}
\pysigline{\sphinxbfcode{\sphinxupquote{W\_AVOIDANCE:}}}
(\sphinxtitleref{float}) ratio of importance of rule of avoidance over the rest.

\end{fulllineitems}



\begin{fulllineitems}
\pysigline{\sphinxbfcode{\sphinxupquote{W\_CENTER:}}}
(\sphinxtitleref{float}) ratio of importance of rule of center over the rest.

\end{fulllineitems}



\begin{fulllineitems}
\pysigline{\sphinxbfcode{\sphinxupquote{W\_COPY:}}}
(\sphinxtitleref{float}) ratio of importance of rule of copy over the rest.

\end{fulllineitems}



\begin{fulllineitems}
\pysigline{\sphinxbfcode{\sphinxupquote{W\_VIEW:}}}
(\sphinxtitleref{float}) ratio of importance of rule of view over the rest.

\end{fulllineitems}



\begin{fulllineitems}
\pysigline{\sphinxbfcode{\sphinxupquote{W\_ATTRACTION:}}}
(\sphinxtitleref{float}) ratio of importance of the attraction of the poins over the rest of rules.

\end{fulllineitems}



\begin{fulllineitems}
\pysigline{\sphinxbfcode{\sphinxupquote{W\_REPULSION:}}}
(\sphinxtitleref{float}) ratio of importance of the repulsion of the poins over the rest of rules.

\end{fulllineitems}



\begin{fulllineitems}
\pysigline{\sphinxbfcode{\sphinxupquote{MU:}}}
(\sphinxtitleref{float}) weight of new velocity vector over current one (used to smooth change of speed and direction).

\end{fulllineitems}


\bigskip 
\bigskip 


\begin{fulllineitems}
\pysigline{\sphinxbfcode{\sphinxupquote{MIN\_DIST:}}}
(\sphinxtitleref{int}) minimum distance that should be between birds (i.e. birds closer than this distance are too close), in pixels.

\end{fulllineitems}



\begin{fulllineitems}
\pysigline{\sphinxbfcode{\sphinxupquote{GROUP\_DIST:}}}
(\sphinxtitleref{int}) distance that determines the boundary of groups of birds (i.e. birds closer than this distance will be considered of the same group), in pixels.

\end{fulllineitems}



\begin{fulllineitems}
\pysigline{\sphinxbfcode{\sphinxupquote{VIEW\_DIST:}}}
(\sphinxtitleref{int}) distance that should be between birds regarding the view rule (i.e. birds that are in the vision area of a bird and closer than this distance are too close), in pixels.

\end{fulllineitems}



\begin{fulllineitems}
\pysigline{\sphinxbfcode{\sphinxupquote{VIEW\_ANGLE:}}}
(\sphinxtitleref{float}) angle that determines the vision area of a bird, in radians.

\end{fulllineitems}


\bigskip 
\bigskip 


\begin{fulllineitems}
\pysigline{\sphinxbfcode{\sphinxupquote{MIN\_DIST\_ATTRACTOR:}}}
(\sphinxtitleref{int}) distance from which attraction points will try to escape from birds (because it will “notice” them), in pixels.

\end{fulllineitems}



\begin{fulllineitems}
\pysigline{\sphinxbfcode{\sphinxupquote{MIN\_DIST\_REPULSOR:}}}
(\sphinxtitleref{int}) distance from which attraction points will try to go towards from birds (because it will “notice” them), in pixels.

\end{fulllineitems}



\begin{fulllineitems}
\pysigline{\sphinxbfcode{\sphinxupquote{GROUP\_DIST\_REPULSOR:}}}
(\sphinxtitleref{int}) distance that determines which birds are withing the group boundary of the repulsion point (so it will try to go towards the center of that group), in pixels.

\end{fulllineitems}


\bigskip 
\bigskip 


\begin{fulllineitems}
\pysigline{\sphinxbfcode{\sphinxupquote{WIDTH:}}}
(\sphinxtitleref{int}) width of screen, in pixels.

\end{fulllineitems}



\begin{fulllineitems}
\pysigline{\sphinxbfcode{\sphinxupquote{HEIGHT:}}}
(\sphinxtitleref{int}) height of screen, in pixels.

\end{fulllineitems}


\bigskip 
\bigskip 


\begin{fulllineitems}
\pysigline{\sphinxbfcode{\sphinxupquote{X\_MIN:}}}
(\sphinxtitleref{int}) minimum value for x coordinate of any bird, in pixels.

\end{fulllineitems}



\begin{fulllineitems}
\pysigline{\sphinxbfcode{\sphinxupquote{X\_MAX:}}}
(\sphinxtitleref{int}) maximum value for x coordinate of any bird, in pixels.

\end{fulllineitems}



\begin{fulllineitems}
\pysigline{\sphinxbfcode{\sphinxupquote{Y\_MIN:}}}
(\sphinxtitleref{int}) minimum value for y coordinate of any bird, in pixels.

\end{fulllineitems}



\begin{fulllineitems}
\pysigline{\sphinxbfcode{\sphinxupquote{Y\_MAX:}}}
(\sphinxtitleref{int}) maximum value for y coordinate of any bird, in pixels.

\end{fulllineitems}



\begin{fulllineitems}
\pysigline{\sphinxbfcode{\sphinxupquote{Z\_MIN:}}}
(\sphinxtitleref{int}) minimum value for z coordinate of any bird, in pixels.

\end{fulllineitems}



\begin{fulllineitems}
\pysigline{\sphinxbfcode{\sphinxupquote{Z\_MAX:}}}
(\sphinxtitleref{int}) maximum value for z coordinate of any bird, in pixels.

\end{fulllineitems}


\bigskip 
\bigskip 


\begin{fulllineitems}
\pysigline{\sphinxbfcode{\sphinxupquote{MIN\_VEL:}}}
(\sphinxtitleref{int}) minimum speed of birds and points of attraction and repulsion.

\end{fulllineitems}



\begin{fulllineitems}
\pysigline{\sphinxbfcode{\sphinxupquote{MAX\_VEL:}}}
(\sphinxtitleref{int}) maximum speed of birds and points of attraction and repulsion.

\end{fulllineitems}


\bigskip 
\bigskip 


\begin{fulllineitems}
\pysigline{\sphinxbfcode{\sphinxupquote{BOUNDARY\_DELTA:}}}
(\sphinxtitleref{float}) threshold considered for the window boundary conditions.

\end{fulllineitems}



\begin{fulllineitems}
\pysigline{\sphinxbfcode{\sphinxupquote{TIME\_DELTA:}}}
(\sphinxtitleref{float}) small interval of time used to update position based on velocity.

\end{fulllineitems}



\begin{fulllineitems}
\pysigline{\sphinxbfcode{\sphinxupquote{DELTA:}}}
(\sphinxtitleref{float}) a small arbitrary float.

\end{fulllineitems}


\bigskip 
\bigskip 


\begin{fulllineitems}
\pysigline{\sphinxbfcode{\sphinxupquote{FPS:}}}
(\sphinxtitleref{int}) determines the speed at which frames are updated.

\end{fulllineitems}


\bigskip
\bigskip 



\begin{fulllineitems}
\pysigline{\sphinxbfcode{\sphinxupquote{ROTATION:}}}
(\sphinxtitleref{int}) determines the speed at which cube rotates when the keys for rotation are pressed.

\end{fulllineitems}



\chapter{bird module}
\label{\detokenize{bird:module-bird}}\label{\detokenize{bird:bird-module}}\label{\detokenize{bird::doc}}\index{module@\spxentry{module}!bird@\spxentry{bird}}\index{bird@\spxentry{bird}!module@\spxentry{module}}\phantomsection\label{\detokenize{bird:module-0}}\index{module@\spxentry{module}!bird@\spxentry{bird}}\index{bird@\spxentry{bird}!module@\spxentry{module}}
Here is defined the structure of a bird in the simulation.

\bigskip 
\bigskip 

\index{Bird (class in bird)@\spxentry{Bird}\spxextra{class in bird}}

\begin{fulllineitems}
\phantomsection\label{\detokenize{bird:bird.Bird}}\pysiglinewithargsret{\sphinxbfcode{\sphinxupquote{class }}\sphinxcode{\sphinxupquote{bird.}}\sphinxbfcode{\sphinxupquote{Bird}}}{\emph{\DUrole{n}{index}\DUrole{p}{:} \DUrole{n}{int}}, \emph{\DUrole{n}{position}\DUrole{p}{:} \DUrole{n}{list}}, \emph{\DUrole{n}{direction}\DUrole{p}{:} \DUrole{n}{list}}, \emph{\DUrole{n}{speed}\DUrole{p}{:} \DUrole{n}{float}}, \emph{\DUrole{n}{type}\DUrole{p}{:} \DUrole{n}{int}}}{}
The class that represents a bird.
\begin{quote}\begin{description}
\item[{Parameters}] \leavevmode\begin{itemize}
\item {} 
\sphinxstyleliteralstrong{\sphinxupquote{index}} (\sphinxstyleliteralemphasis{\sphinxupquote{int}}) \textendash{} index that identifies bird.

\item {} 
\sphinxstyleliteralstrong{\sphinxupquote{position}} (\sphinxstyleliteralemphasis{\sphinxupquote{list}}) \textendash{} coordinates (x,y,z) of bird.

\item {} 
\sphinxstyleliteralstrong{\sphinxupquote{direction}} (\sphinxstyleliteralemphasis{\sphinxupquote{list}}) \textendash{} direction of bird’s velocity vector with coordinates (x,y,z), as a unit vector.

\item {} 
\sphinxstyleliteralstrong{\sphinxupquote{speed}} (\sphinxstyleliteralemphasis{\sphinxupquote{float}}) \textendash{} module of bird’s velocity vector.

\item {} 
\sphinxstyleliteralstrong{\sphinxupquote{type}} (\sphinxstyleliteralemphasis{\sphinxupquote{int}}) \textendash{} the type of object that the instance represents. Value 1 for bird, \sphinxhyphen{}1 for attraction point, \sphinxhyphen{}2 for repulsion point.

\end{itemize}

\end{description}\end{quote}

\bigskip 
\bigskip 

\begin{fulllineitems}
\phantomsection\label{\detokenize{bird:bird.Bird.attraction}}\pysiglinewithargsret{\sphinxbfcode{\sphinxupquote{attraction}}}{\emph{\DUrole{n}{attraction\_points}}}{}
Go towards attraction points.
\begin{quote}\begin{description}
\item[{Parameters}] \leavevmode
\sphinxstyleliteralstrong{\sphinxupquote{attraction\_points}} (\sphinxstyleliteralemphasis{\sphinxupquote{list}}) \textendash{} list of coordinates of the attraction points (see \sphinxcode{\sphinxupquote{ATTRACTION\_POINTS}} in {\hyperref[\detokenize{parameters:module-0}]{\sphinxcrossref{\sphinxcode{\sphinxupquote{parameters}}}}}).

\item[{Returns}] \leavevmode
velocity vector that responds to the attraction of the corresponding points.

\item[{Return type}] \leavevmode
\textit{list}

\end{description}\end{quote}

\bigskip 
\bigskip 

\end{fulllineitems}

\index{avoidance() (bird.Bird method)@\spxentry{avoidance()}\spxextra{bird.Bird method}}

\begin{fulllineitems}
\phantomsection\label{\detokenize{bird:bird.Bird.avoidance}}\pysiglinewithargsret{\sphinxbfcode{\sphinxupquote{avoidance}}}{\emph{\DUrole{n}{neighbours}\DUrole{p}{:} \DUrole{n}{list}}}{}
Separate bird from neighbours that are too close.
\begin{quote}\begin{description}
\item[{Parameters}] \leavevmode
\sphinxstyleliteralstrong{\sphinxupquote{neighbours}} (\sphinxstyleliteralemphasis{\sphinxupquote{list}}) \textendash{} birds that are closer to the bird than the minimum distance (see \sphinxcode{\sphinxupquote{MIN\_DIST}} in {\hyperref[\detokenize{parameters:module-0}]{\sphinxcrossref{\sphinxcode{\sphinxupquote{parameters}}}}}). Birds are represented as instances of the {\hyperref[\detokenize{bird:bird.Bird}]{\sphinxcrossref{\sphinxcode{\sphinxupquote{bird.Bird}}}}} class.

\item[{Returns}] \leavevmode
velocity vector that responds to the Avoidance rule.

\item[{Return type}] \leavevmode
\textit{list}

\end{description}\end{quote}

\bigskip 
\bigskip 

\end{fulllineitems}

\index{center() (bird.Bird method)@\spxentry{center()}\spxextra{bird.Bird method}}

\begin{fulllineitems}
\phantomsection\label{\detokenize{bird:bird.Bird.center}}\pysiglinewithargsret{\sphinxbfcode{\sphinxupquote{center}}}{\emph{\DUrole{n}{group\_birds}\DUrole{p}{:} \DUrole{n}{list}}}{}
Seek cohesion with other bird’s positions.
Bird will change direction to move toward the average position of all birds.
\begin{quote}\begin{description}
\item[{Parameters}] \leavevmode
\sphinxstyleliteralstrong{\sphinxupquote{group\_birds}} (\sphinxstyleliteralemphasis{\sphinxupquote{list}}) \textendash{} birds that are closer to the bird than the group boundary distance (see \sphinxcode{\sphinxupquote{GROUP\_DIST}} in {\hyperref[\detokenize{parameters:module-0}]{\sphinxcrossref{\sphinxcode{\sphinxupquote{parameters}}}}}). Birds are represented as instances of the {\hyperref[\detokenize{bird:bird.Bird}]{\sphinxcrossref{\sphinxcode{\sphinxupquote{bird.Bird}}}}} class.

\item[{Returns}] \leavevmode
velocity vector that responds to the Center rule.

\item[{Return type}] \leavevmode
\textit{list}

\end{description}\end{quote}

\bigskip 
\bigskip 

\end{fulllineitems}

\index{copy() (bird.Bird method)@\spxentry{copy()}\spxextra{bird.Bird method}}

\begin{fulllineitems}
\phantomsection\label{\detokenize{bird:bird.Bird.copy}}\pysiglinewithargsret{\sphinxbfcode{\sphinxupquote{copy}}}{\emph{\DUrole{n}{group\_birds}\DUrole{p}{:} \DUrole{n}{list}}}{}
Seek cohesion with other bird’s directions (average direction).
\begin{quote}\begin{description}
\item[{Parameters}] \leavevmode
\sphinxstyleliteralstrong{\sphinxupquote{group\_birds}} (\sphinxstyleliteralemphasis{\sphinxupquote{list}}) \textendash{} birds that are closer to the bird than the group boundary distance (see \sphinxcode{\sphinxupquote{GROUP\_DIST}} in {\hyperref[\detokenize{parameters:module-0}]{\sphinxcrossref{\sphinxcode{\sphinxupquote{parameters}}}}}). Birds are represented as instances of the {\hyperref[\detokenize{bird:bird.Bird}]{\sphinxcrossref{\sphinxcode{\sphinxupquote{bird.Bird}}}}} class.

\item[{Returns}] \leavevmode
velocity vector that responds to the Copy rule.

\item[{Return type}] \leavevmode
\textit{list}

\end{description}\end{quote}

\bigskip 
\bigskip 

\end{fulllineitems}

\index{repulsion() (bird.Bird method)@\spxentry{repulsion()}\spxextra{bird.Bird method}}

\begin{fulllineitems}
\phantomsection\label{\detokenize{bird:bird.Bird.repulsion}}\pysiglinewithargsret{\sphinxbfcode{\sphinxupquote{repulsion}}}{\emph{\DUrole{n}{repulsion\_points}}}{}
Go towards repulsion points.
\begin{quote}\begin{description}
\item[{Parameters}] \leavevmode
\sphinxstyleliteralstrong{\sphinxupquote{repulsion\_points}} (\sphinxstyleliteralemphasis{\sphinxupquote{list}}) \textendash{} list of coordinates of the repulsion points (see \sphinxcode{\sphinxupquote{REPULSION\_POINTS}} in {\hyperref[\detokenize{parameters:module-0}]{\sphinxcrossref{\sphinxcode{\sphinxupquote{parameters}}}}}).

\item[{Returns}] \leavevmode
velocity vector that responds to the repulsion of the corresponding points.

\item[{Return type}] \leavevmode
\textit{list}

\end{description}\end{quote}

\bigskip 
\bigskip 

\end{fulllineitems}

\index{update() (bird.Bird method)@\spxentry{update()}\spxextra{bird.Bird method}}

\begin{fulllineitems}
\phantomsection\label{\detokenize{bird:bird.Bird.update}}\pysiglinewithargsret{\sphinxbfcode{\sphinxupquote{update}}}{\emph{\DUrole{n}{close\_neighbours}}, \emph{\DUrole{n}{group\_birds}}, \emph{\DUrole{n}{attraction\_points}}, \emph{\DUrole{n}{repulsion\_points}}}{}
Updates direction, speed and position of bird, considering all rules, and the attraction and repulsion points.
\begin{quote}\begin{description}
\item[{Parameters}] \leavevmode\begin{itemize}
\item {} 
\sphinxstyleliteralstrong{\sphinxupquote{close\_neighbours}} (\sphinxstyleliteralemphasis{\sphinxupquote{list}}) \textendash{} birds that are closer to the bird than the minimum distance (see \sphinxcode{\sphinxupquote{MIN\_DIST}} in {\hyperref[\detokenize{parameters:module-0}]{\sphinxcrossref{\sphinxcode{\sphinxupquote{parameters}}}}}). Birds are represented as instances of the {\hyperref[\detokenize{bird:bird.Bird}]{\sphinxcrossref{\sphinxcode{\sphinxupquote{bird.Bird}}}}} class.

\item {} 
\sphinxstyleliteralstrong{\sphinxupquote{group\_birds}} (\sphinxstyleliteralemphasis{\sphinxupquote{list}}) \textendash{} birds that are closer to the bird than the group boundary distance (see \sphinxcode{\sphinxupquote{GROUP\_DIST}} in {\hyperref[\detokenize{parameters:module-0}]{\sphinxcrossref{\sphinxcode{\sphinxupquote{parameters}}}}}). Birds are represented as instances of the {\hyperref[\detokenize{bird:bird.Bird}]{\sphinxcrossref{\sphinxcode{\sphinxupquote{bird.Bird}}}}} class.

\item {} 
\sphinxstyleliteralstrong{\sphinxupquote{attraction\_points}} (\sphinxstyleliteralemphasis{\sphinxupquote{list}}) \textendash{} list of coordinates of the attraction points (see \sphinxcode{\sphinxupquote{ATTRACTION\_POINTS}} in {\hyperref[\detokenize{parameters:module-0}]{\sphinxcrossref{\sphinxcode{\sphinxupquote{parameters}}}}}).

\item {} 
\sphinxstyleliteralstrong{\sphinxupquote{repulsion\_points}} (\sphinxstyleliteralemphasis{\sphinxupquote{list}}) \textendash{} list of coordinates of the repulsion points (see \sphinxcode{\sphinxupquote{REPULSION\_POINTS}} in {\hyperref[\detokenize{parameters:module-0}]{\sphinxcrossref{\sphinxcode{\sphinxupquote{parameters}}}}}).

\end{itemize}

\end{description}\end{quote}

\bigskip 
\bigskip 

\end{fulllineitems}

\index{updateAttractor() (bird.Bird method)@\spxentry{updateAttractor()}\spxextra{bird.Bird method}}

\begin{fulllineitems}
\phantomsection\label{\detokenize{bird:bird.Bird.updateAttractor}}\pysiglinewithargsret{\sphinxbfcode{\sphinxupquote{updateAttractor}}}{\emph{\DUrole{n}{all\_birds}}}{}
Updates direction, speed and position of the attractor points. 
They will avoid the birds that are closer than a minimum distance (see \sphinxcode{\sphinxupquote{MIN\_DIST\_ATTRACTOR}} in {\hyperref[\detokenize{parameters:module-0}]{\sphinxcrossref{\sphinxcode{\sphinxupquote{parameters}}}}}).
\begin{quote}\begin{description}
\item[{Parameters}] \leavevmode
\sphinxstyleliteralstrong{\sphinxupquote{all\_birds}} (\sphinxstyleliteralemphasis{\sphinxupquote{list}}) \textendash{} all the birds in the simulation, represented as instances of the {\hyperref[\detokenize{bird:bird.Bird}]{\sphinxcrossref{\sphinxcode{\sphinxupquote{bird.Bird}}}}} class.

\end{description}\end{quote}

\bigskip 
\bigskip 

\end{fulllineitems}

\index{updatePos() (bird.Bird method)@\spxentry{updatePos()}\spxextra{bird.Bird method}}

\begin{fulllineitems}
\phantomsection\label{\detokenize{bird:bird.Bird.updatePos}}\pysiglinewithargsret{\sphinxbfcode{\sphinxupquote{updatePos}}}{\emph{\DUrole{n}{diff\_time}}}{}
Update bird’s position using speed and direction.
Takes into consideration boundary conditions.
\begin{quote}\begin{description}
\item[{Parameters}] \leavevmode
\sphinxstyleliteralstrong{\sphinxupquote{diff\_time}} (\sphinxstyleliteralemphasis{\sphinxupquote{float}}) \textendash{} small interval of time used to update position based on velocity.

\end{description}\end{quote}

\bigskip 
\bigskip 

\end{fulllineitems}

\index{updateRepulsor() (bird.Bird method)@\spxentry{updateRepulsor()}\spxextra{bird.Bird method}}

\begin{fulllineitems}
\phantomsection\label{\detokenize{bird:bird.Bird.updateRepulsor}}\pysiglinewithargsret{\sphinxbfcode{\sphinxupquote{updateRepulsor}}}{\emph{\DUrole{n}{all\_birds}}}{}
Updates direction, speed and position of the repulsion points. 
They will go towards the birds that are closer than a minimum distance (see \sphinxcode{\sphinxupquote{MIN\_DIST\_REPULSOR}} in {\hyperref[\detokenize{parameters:module-0}]{\sphinxcrossref{\sphinxcode{\sphinxupquote{parameters}}}}}).
They will also go towards the center of the group of birds that are closer than a group boundary distance (see \sphinxcode{\sphinxupquote{GROUP\_DIST\_REPULSOR}} in {\hyperref[\detokenize{parameters:module-0}]{\sphinxcrossref{\sphinxcode{\sphinxupquote{parameters}}}}}).
\begin{quote}\begin{description}
\item[{Parameters}] \leavevmode
\sphinxstyleliteralstrong{\sphinxupquote{all\_birds}} (\sphinxstyleliteralemphasis{\sphinxupquote{list}}) \textendash{} all the birds in the simulation, represented as instances of the {\hyperref[\detokenize{bird:bird.Bird}]{\sphinxcrossref{\sphinxcode{\sphinxupquote{bird.Bird}}}}} class.

\end{description}\end{quote}

\bigskip 
\bigskip 

\end{fulllineitems}

\index{view() (bird.Bird method)@\spxentry{view()}\spxextra{bird.Bird method}}

\begin{fulllineitems}
\phantomsection\label{\detokenize{bird:bird.Bird.view}}\pysiglinewithargsret{\sphinxbfcode{\sphinxupquote{view}}}{\emph{\DUrole{n}{group\_birds}\DUrole{p}{:} \DUrole{n}{list}}}{}
Move if there is another bird in area of view.
\begin{quote}\begin{description}
\item[{Parameters}] \leavevmode
\sphinxstyleliteralstrong{\sphinxupquote{group\_birds}} (\sphinxstyleliteralemphasis{\sphinxupquote{list}}) \textendash{} birds that are closer to the bird than the group boundary distance (see \sphinxcode{\sphinxupquote{GROUP\_DIST}} in {\hyperref[\detokenize{parameters:module-0}]{\sphinxcrossref{\sphinxcode{\sphinxupquote{parameters}}}}}). Birds are represented as instances of the {\hyperref[\detokenize{bird:bird.Bird}]{\sphinxcrossref{\sphinxcode{\sphinxupquote{bird.Bird}}}}} class.

\item[{Returns}] \leavevmode
velocity vector that responds to the View rule.

\item[{Return type}] \leavevmode
\textit{list}

\end{description}\end{quote}

\bigskip 
\bigskip 

\end{fulllineitems}


\end{fulllineitems}



\chapter{initialize\_birds module}
\label{\detokenize{initialize_birds:module-initialize_birds}}\label{\detokenize{initialize_birds:initialize-birds-module}}\label{\detokenize{initialize_birds::doc}}\index{module@\spxentry{module}!initialize\_birds@\spxentry{initialize\_birds}}\index{initialize\_birds@\spxentry{initialize\_birds}!module@\spxentry{module}}\phantomsection\label{\detokenize{initialize_birds:module-0}}\index{module@\spxentry{module}!initialize\_birds@\spxentry{initialize\_birds}}\index{initialize\_birds@\spxentry{initialize\_birds}!module@\spxentry{module}}
Initialization of birds.

\bigskip 
\bigskip 

\index{generateAttractionPoints() (in module initialize\_birds)@\spxentry{generateAttractionPoints()}\spxextra{in module initialize\_birds}}

\begin{fulllineitems}
\phantomsection\label{\detokenize{initialize_birds:initialize_birds.generateAttractionPoints}}\pysiglinewithargsret{\sphinxcode{\sphinxupquote{initialize\_birds.}}\sphinxbfcode{\sphinxupquote{generateAttractionPoints}}}{}{}
Generates a list of attraction points.
\begin{quote}\begin{description}
\item[{Returns}] \leavevmode
list of instances of the class {\hyperref[\detokenize{bird:bird.Bird}]{\sphinxcrossref{\sphinxcode{\sphinxupquote{bird.Bird}}}}}, with the attribute \sphinxcode{\sphinxupquote{type}} assigned to \sphinxhyphen{}1 (which represents an Attraction Point).

\item[{Return type}] \leavevmode
\textit{list}

\end{description}\end{quote}

\bigskip 
\bigskip 

\end{fulllineitems}

\index{generateBirds() (in module initialize\_birds)@\spxentry{generateBirds()}\spxextra{in module initialize\_birds}}

\begin{fulllineitems}
\phantomsection\label{\detokenize{initialize_birds:initialize_birds.generateBirds}}\pysiglinewithargsret{\sphinxcode{\sphinxupquote{initialize\_birds.}}\sphinxbfcode{\sphinxupquote{generateBirds}}}{}{}
Generates a list of birds. Positions and velocity are random.
\begin{quote}\begin{description}
\item[{Returns}] \leavevmode
list of instances of the class {\hyperref[\detokenize{bird:bird.Bird}]{\sphinxcrossref{\sphinxcode{\sphinxupquote{bird.Bird}}}}}.

\item[{Return type}] \leavevmode
\textit{list}

\end{description}\end{quote}

\bigskip 
\bigskip 

\end{fulllineitems}

\index{generateRepulsionPoints() (in module initialize\_birds)@\spxentry{generateRepulsionPoints()}\spxextra{in module initialize\_birds}}

\begin{fulllineitems}
\phantomsection\label{\detokenize{initialize_birds:initialize_birds.generateRepulsionPoints}}\pysiglinewithargsret{\sphinxcode{\sphinxupquote{initialize\_birds.}}\sphinxbfcode{\sphinxupquote{generateRepulsionPoints}}}{}{}
Generates a list of repulsion points.
\begin{quote}\begin{description}
\item[{Returns}] \leavevmode
list of instances of the class {\hyperref[\detokenize{bird:bird.Bird}]{\sphinxcrossref{\sphinxcode{\sphinxupquote{bird.Bird}}}}}, with the attribute \sphinxcode{\sphinxupquote{type}} assigned to \sphinxhyphen{}2 (which represents an Repulsion Point).

\item[{Return type}] \leavevmode
\textit{list}

\end{description}\end{quote}

\bigskip 
\bigskip 

\end{fulllineitems}



\chapter{graphics module}
\label{\detokenize{graphics:module-graphics}}\label{\detokenize{graphics:graphics-module}}\label{\detokenize{graphics::doc}}\index{module@\spxentry{module}!graphics@\spxentry{graphics}}\index{graphics@\spxentry{graphics}!module@\spxentry{module}}\phantomsection\label{\detokenize{graphics:module-0}}\index{module@\spxentry{module}!graphics@\spxentry{graphics}}\index{graphics@\spxentry{graphics}!module@\spxentry{module}}
Functions used to render graphics.

\bigskip 
\bigskip 

\index{draw\_circle() (in module graphics)@\spxentry{draw\_circle()}\spxextra{in module graphics}}

\begin{fulllineitems}
\phantomsection\label{\detokenize{graphics:graphics.draw_circle}}\pysiglinewithargsret{\sphinxcode{\sphinxupquote{graphics.}}\sphinxbfcode{\sphinxupquote{draw\_circle}}}{\emph{\DUrole{n}{position}}, \emph{\DUrole{n}{color}}, \emph{\DUrole{n}{radius}\DUrole{o}{=}\DUrole{default_value}{10}}, \emph{\DUrole{n}{side\_num}\DUrole{o}{=}\DUrole{default_value}{10}}}{}
Draws a circle of a given color.
\begin{quote}\begin{description}
\item[{Parameters}] \leavevmode\begin{itemize}
\item {} 
\sphinxstyleliteralstrong{\sphinxupquote{position}} (\sphinxstyleliteralemphasis{\sphinxupquote{list}}) \textendash{} coordinates of the circle’s center point.

\item {} 
\sphinxstyleliteralstrong{\sphinxupquote{color}} (\sphinxstyleliteralemphasis{\sphinxupquote{str}}) \textendash{} the color of the circle. For example: ‘red’, ‘green’.

\item {} 
\sphinxstyleliteralstrong{\sphinxupquote{radius}} (\sphinxstyleliteralemphasis{\sphinxupquote{int}}\sphinxstyleliteralemphasis{\sphinxupquote{, }}\sphinxstyleliteralemphasis{\sphinxupquote{optional}}) \textendash{} radius of the circle (in pixels), defaults to 10.

\item {} 
\sphinxstyleliteralstrong{\sphinxupquote{side\_num}} (\sphinxstyleliteralemphasis{\sphinxupquote{int}}\sphinxstyleliteralemphasis{\sphinxupquote{, }}\sphinxstyleliteralemphasis{\sphinxupquote{optional}}) \textendash{} radius of the circle (in pixels), defaults to 10.

\end{itemize}

\end{description}\end{quote}

\bigskip 
\bigskip 

\end{fulllineitems}

\index{draw\_cone() (in module graphics)@\spxentry{draw\_cone()}\spxextra{in module graphics}}

\begin{fulllineitems}
\phantomsection\label{\detokenize{graphics:graphics.draw_cone}}\pysiglinewithargsret{\sphinxcode{\sphinxupquote{graphics.}}\sphinxbfcode{\sphinxupquote{draw\_cone}}}{\emph{\DUrole{n}{pos}}, \emph{\DUrole{n}{direction}}, \emph{\DUrole{n}{radius}}, \emph{\DUrole{n}{height}}, \emph{\DUrole{n}{slices}\DUrole{o}{=}\DUrole{default_value}{7}}, \emph{\DUrole{n}{stacks}\DUrole{o}{=}\DUrole{default_value}{1}}}{}
Draws a black cone inisde the container.
\begin{quote}\begin{description}
\item[{Parameters}] \leavevmode\begin{itemize}
\item {} 
\sphinxstyleliteralstrong{\sphinxupquote{pos}} (\sphinxstyleliteralemphasis{\sphinxupquote{list}}) \textendash{} coordinates of the cone’s head vertex.

\item {} 
\sphinxstyleliteralstrong{\sphinxupquote{direction}} (\sphinxstyleliteralemphasis{\sphinxupquote{list}}) \textendash{} unitary vector that represents the direction in wich the cone is pointed.

\item {} 
\sphinxstyleliteralstrong{\sphinxupquote{radius}} (\sphinxstyleliteralemphasis{\sphinxupquote{int}}) \textendash{} radius of the cones base, in pixels.

\item {} 
\sphinxstyleliteralstrong{\sphinxupquote{height}} (\sphinxstyleliteralemphasis{\sphinxupquote{int}}) \textendash{} height of the cone, in pixels.

\item {} 
\sphinxstyleliteralstrong{\sphinxupquote{slices}} (\sphinxstyleliteralemphasis{\sphinxupquote{int}}\sphinxstyleliteralemphasis{\sphinxupquote{, }}\sphinxstyleliteralemphasis{\sphinxupquote{optional}}) \textendash{} Number of slices that will determine the cone’s shape in the graphics, defaults to 7.

\item {} 
\sphinxstyleliteralstrong{\sphinxupquote{stacks}} (\sphinxstyleliteralemphasis{\sphinxupquote{int}}\sphinxstyleliteralemphasis{\sphinxupquote{, }}\sphinxstyleliteralemphasis{\sphinxupquote{optional}}) \textendash{} Number of stacks that will determine the cone’s shape in the graphics, defaults to 1.

\end{itemize}

\end{description}\end{quote}

\bigskip 
\bigskip 

\end{fulllineitems}

\index{draw\_container() (in module graphics)@\spxentry{draw\_container()}\spxextra{in module graphics}}

\begin{fulllineitems}
\phantomsection\label{\detokenize{graphics:graphics.draw_container}}\pysiglinewithargsret{\sphinxcode{\sphinxupquote{graphics.}}\sphinxbfcode{\sphinxupquote{draw\_container}}}{}{}
Draws 2D square or 3D cube that contains birds.

\bigskip 
\bigskip 

\end{fulllineitems}

\index{draw\_sphere() (in module graphics)@\spxentry{draw\_sphere()}\spxextra{in module graphics}}

\begin{fulllineitems}
\phantomsection\label{\detokenize{graphics:graphics.draw_sphere}}\pysiglinewithargsret{\sphinxcode{\sphinxupquote{graphics.}}\sphinxbfcode{\sphinxupquote{draw\_sphere}}}{\emph{\DUrole{n}{position}}, \emph{\DUrole{n}{color}}, \emph{\DUrole{n}{r}\DUrole{o}{=}\DUrole{default_value}{10}}, \emph{\DUrole{n}{lats}\DUrole{o}{=}\DUrole{default_value}{10}}, \emph{\DUrole{n}{longs}\DUrole{o}{=}\DUrole{default_value}{10}}}{}
Draws a sphere of a given color.
\begin{quote}\begin{description}
\item[{Parameters}] \leavevmode\begin{itemize}
\item {} 
\sphinxstyleliteralstrong{\sphinxupquote{position}} (\sphinxstyleliteralemphasis{\sphinxupquote{list}}) \textendash{} coordinates of the sphere’s center point.

\item {} 
\sphinxstyleliteralstrong{\sphinxupquote{color}} (\sphinxstyleliteralemphasis{\sphinxupquote{str}}) \textendash{} the color of the sphere. For example: ‘red’, ‘green’

\item {} 
\sphinxstyleliteralstrong{\sphinxupquote{r}} (\sphinxstyleliteralemphasis{\sphinxupquote{int}}\sphinxstyleliteralemphasis{\sphinxupquote{, }}\sphinxstyleliteralemphasis{\sphinxupquote{optional}}) \textendash{} radius of the sphere (in pixels), defaults to 10.

\item {} 
\sphinxstyleliteralstrong{\sphinxupquote{lats}} (\sphinxstyleliteralemphasis{\sphinxupquote{int}}\sphinxstyleliteralemphasis{\sphinxupquote{, }}\sphinxstyleliteralemphasis{\sphinxupquote{optional}}) \textendash{} number of lats that will determine the sphere’s shape in the graphics, defaults to 10.

\item {} 
\sphinxstyleliteralstrong{\sphinxupquote{longs}} (\sphinxstyleliteralemphasis{\sphinxupquote{int}}\sphinxstyleliteralemphasis{\sphinxupquote{, }}\sphinxstyleliteralemphasis{\sphinxupquote{optional}}) \textendash{} number of longs that will determine the sphere’s shape in the graphics, defaults to 10.

\end{itemize}

\end{description}\end{quote}

\bigskip 
\bigskip 

\end{fulllineitems}

\index{draw\_triangle() (in module graphics)@\spxentry{draw\_triangle()}\spxextra{in module graphics}}

\begin{fulllineitems}
\phantomsection\label{\detokenize{graphics:graphics.draw_triangle}}\pysiglinewithargsret{\sphinxcode{\sphinxupquote{graphics.}}\sphinxbfcode{\sphinxupquote{draw\_triangle}}}{\emph{\DUrole{n}{head}}, \emph{\DUrole{n}{tail\_vertex1}}, \emph{\DUrole{n}{tail\_vertex2}}}{}
Draws a black triangle inisde the container.
\begin{quote}\begin{description}
\item[{Parameters}] \leavevmode\begin{itemize}
\item {} 
\sphinxstyleliteralstrong{\sphinxupquote{head}} (\sphinxstyleliteralemphasis{\sphinxupquote{list}}) \textendash{} coordinates of the triangle’s head vertex.

\item {} 
\sphinxstyleliteralstrong{\sphinxupquote{tail\_vertex1}} (\sphinxstyleliteralemphasis{\sphinxupquote{list}}) \textendash{} coordinates of a tail’s vertex.

\item {} 
\sphinxstyleliteralstrong{\sphinxupquote{tail\_vertex2}} (\sphinxstyleliteralemphasis{\sphinxupquote{list}}) \textendash{} coordinates of the other tail’s vertex.

\end{itemize}

\end{description}\end{quote}

\bigskip 
\bigskip 

\end{fulllineitems}

\index{initialize\_window() (in module graphics)@\spxentry{initialize\_window()}\spxextra{in module graphics}}

\begin{fulllineitems}
\phantomsection\label{\detokenize{graphics:graphics.initialize_window}}\pysiglinewithargsret{\sphinxcode{\sphinxupquote{graphics.}}\sphinxbfcode{\sphinxupquote{initialize\_window}}}{}{}
Initializes window where simulation will be shown when running the program.

\bigskip 
\bigskip 

\end{fulllineitems}



\chapter{main module}
\label{\detokenize{main:module-main}}\label{\detokenize{main:main-module}}\label{\detokenize{main::doc}}\index{module@\spxentry{module}!main@\spxentry{main}}\index{main@\spxentry{main}!module@\spxentry{module}}\phantomsection\label{\detokenize{main:module-0}}\index{module@\spxentry{module}!main@\spxentry{main}}\index{main@\spxentry{main}!module@\spxentry{module}}
File where simulation is runned.

\bigskip 
\bigskip 

\index{main() (in module main)@\spxentry{main()}\spxextra{in module main}}

\begin{fulllineitems}
\phantomsection\label{\detokenize{main:main.main}}\pysiglinewithargsret{\sphinxcode{\sphinxupquote{main.}}\sphinxbfcode{\sphinxupquote{main}}}{}{}
Function that has to be executed to run the simulation.

\bigskip 
\bigskip 

\end{fulllineitems}


\end{document}